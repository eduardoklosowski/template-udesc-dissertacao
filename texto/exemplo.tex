\chapter{Exemplo}\label{texto:exemplo}

Este capítulo é um exemplo de utilização deste template, apresentando como utilizar os elementos oferecidos.

Para mais informações veja a documentação dos pacotes:

\begin{incisos}
    \item \url{https://www.ctan.org/pkg/memoir}
    \item \url{https://www.ctan.org/pkg/abntex2}
\end{incisos}

\section{Citação}\label{texto:exemplo:citacao}

\subsection{Citação curta}\label{texto:exemplo:citacao:curta}

A citação curta vai no decorrer do texto entre aspas, como ``exemplo de citação curta'' \cite[p.~20]{exemplodelivro}. Ou como citação indireta por \citeonline{exemplodelivro}.

Resumidamente, utilize:

\begin{verbatim}
    \cite{exemplodelivro}  ->  (AUTOR, 2020)
    \cite[p.~20]{exemplodelivro}  ->  (AUTOR, 2020, p. 20)
    \citeonline{exemplodelivro}  ->  AUTOR (2020)
\end{verbatim}

\subsection{Citação longa}\label{texto:exemplo:citacao:longa}

\begin{citacao}
    Exemplo de citação longa, transcrevendo o texto dentro de um contexto \texttt{citacao} do Latex \cite[p.~20-21]{exemplodelivro}.
\end{citacao}

Ela também pode ser feita em inglês:

\begin{citacao}[english]
    Text in English language in italic with correct hyphenation \cite[p.~20]{exemplodelivro}.
\end{citacao}

\subsection{Apud}\label{texto:exemplo:citacao:apud}

Este é um exemplo de apud: \apud{exemplodelivro}{exemplodeartigo}, ou \apudonline{exemplodelivro}{exemplodeproceedings}.

\section{Texto em inglês}\label{texto:exemplo:textoingles}

\begin{otherlanguage*}{english}
    English text that can automatically receive hyphens if needed.
\end{otherlanguage*}

\section{Figura}\label{texto:exemplo:figura}

A \autoref{fig:grafico} mostra um gráfico, que também pode estar lado a lado com outro, como \autoref{fig:grafico1} e \autoref{fig:grafico2}.

\begin{figure}[htb]
    \caption{Gráfico de exemplo}
    \label{fig:grafico}
    \centering
    \includegraphics[width=.5\linewidth]{fig/grafico.pdf}
    \legend{Fonte: Elaborada pelo autor, 2020.}
\end{figure}

\begin{figure}[htb]
 \centering
  \begin{minipage}{0.45\textwidth}
    \caption{Gráfico 1}
    \label{fig:grafico1}
    \centering
    \includegraphics[scale=0.15]{fig/grafico.pdf}
    \legend{Fonte: \citeonline[p.~20]{exemplodelivro}.}
  \end{minipage}
  \hfill
  \begin{minipage}{0.45\textwidth}
    \caption{Gráfico 2}
    \label{fig:grafico2}
    \centering
    \includegraphics[scale=0.15]{fig/grafico.pdf}
    \legend{Fonte: \citeonline[p.~20]{exemplodelivro}.}
  \end{minipage}
\end{figure}

\section{Tabela}\label{texto:exemplo:tabela}

A \autoref{tab:exemplo} é um exemplo.

\begin{table}[htb]
    \caption{Tabela de exemplo}
    \label{tab:exemplo}
    \centering
    \ABNTEXfontereduzida
    \begin{tabular}{l|l}
        \hline
        \textbf{Indivíduo} & \textbf{Resultado} \\
        \hline
        A & 5 \\
        \hline
        B & 7 \\
        \hline
    \end{tabular}
    \legend{Fonte: \citeonline{exemplodelivro}.}
\end{table}

\section{Expressões matemáticas}\label{texto:exemplo:expressoesmatematicas}

Use o contexto \texttt{equation} para escrever
expressões matemáticas, conforme o exemplo da \autoref{eq:exemplo}.

\begin{equation}
    \label{eq:exemplo}
    \Delta s  = v \cdot t
\end{equation}

\section{Nota de rodapé}\label{texto:exemplo:notarodape}

Este texto possui uma nota de rodapé\footnote{Esta é uma nota de rodapé.}.

\section{Siglas}\label{texto:exemplo:siglas}

As siglas devem ser definidas no arquivo \texttt{pretexto/siglas.tex}, depois podem ser utilizadas com \ac{OV} para singular, ou \acp{OV} para plural. Na primeira vez o nome aparecerá por extenso, nas demais apenas a sigla.

Outros exemplo: \ac{SDN} e \ac{SGBD}.

Veja mais em: \url{https://www.ctan.org/pkg/acronym}.

\section{Seção secundária}\label{texto:exemplo:secao-secundaria}

Exemplo de seção secundária.

\subsection{Seção terciária}\label{texto:exemplo:secao-secundaria:secao-terciaria}

Exemplo de seção terciária.

\subsubsection{Seção quaternária}\label{texto:exemplo:secao-secundaria:secao-terciaria:secao-quaternaria}

Exemplo de seção quaternária.

\subsubsubsection{Seção quinária}\label{texto:exemplo:secao-secundaria:secao-terciaria:secao-quaternaria:secao-quinaria}

Exemplo de seção quinária.

E qualquer seção pode ser referenciada com nestes exemplos: \autoref{texto:exemplo} e \autoref{texto:exemplo:secao-secundaria:secao-terciaria:secao-quaternaria:secao-quinaria}.

\section{Enumeração}\label{texto:exemplo:enumeracao}

\begin{alineas}
    \item Itens enumerados;
    \item Sub-itens:
    \begin{alineas}
        \item 1;
        \item 2.
    \end{alineas}
    \item O último termina com ponto.
\end{alineas}
